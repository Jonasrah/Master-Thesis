- Conceptual models in games can be the game world or the theme of the game. e.g. a game set in an old Aztec-like temple would give the player a mental model of there possibly would be residing traps and puzzles, like they have learned from for example Indiana Jones. Totten. Rollings
``object-oriented analysis, or data structure or database design.
For example, when the designers of a hotel registration system debate
whether a reservation should be recorded as a feature of a room or of a
day or of a person, they are designing the “ontology” of their system. At
the same time, “ontology” is also sometimes used to refer to the ele-
ments of a user’s conceptual model—the model either of the user’s own
work or of the system’s operation. Users’ understandings are a matter of
some significance to system designers, who needs to design a system
model that will fit “the user’s ontology.”'' dourish 130.

- Keyboard Sport uses natural mapping

- Norman's natural mapping and best, second-best and third-best mapping is not connected in any way. It is unclear what the difference between a good utilisation of mapping and natural mapping and since 'best mapping' is where controls are mounted directly to the controlled, natural mapping only qualifies as either second-best or third-best.

- The way the scroll wheel is interacted with becomes a carrier of meaning in the sense that one can scroll rapidly to gain speed or slow for precision.

- Action-perception loop == Feedback loop?

- Logical constraints can provide coupling: A room with a door and a button. Logical conclusion: Button operates door. Connected to decoupling.

- Framework can also be used to decouple unintended action-perception loops.

 - Affordance addition: "An affordance is a three-way relationship between the environment, the organism, and an activity" \cite[p. 118]{dourish}.

 - Dourish and meaning:
  - A user's ontology == a user's conceptual model (p. 129).
  - So embodied interaction is not possible for the player, but what about the agent that the player controls? Schutz intersubjectivity.

Three solutions:
 - The advanced model with dual focus
 - A QA questionnaire about what information the player experience
 - A simplified model for practical use to see how the information from the questionnaire flows


TOPICS FOR THE DEFENSE:
- The method when used can lead to rethinking of the game mechanic which leads to game mechanics designed around challenges which I think is great.
- Direction is more complex than it implies. Mario jumping by pushing a bottom down is actually coupled because it is opposite. Same for prototype.
- Problems with expression: In feedforward a red turbo button only informs expressively because of the other buttons being black.




 Questions:

  - What action would you invoke in the physical world to overcome the challenge?
   - Why?
   MW: Scroll Fast Backwards
   M: Scrolling forward
   K: Scrolling forward
   I: Scroll with the wheel, or use it as a ramp
   J: Scroll back and forth

    - When would you apply this action and when would you stop?
     - Why?
   MW: Go backwards to gain momentum and stop once airborne
   M: Apply until to the top of the wheel and then stop.
   K: Continuous scrolling until top, then stop
   I: Continuous scrolling until top, then stop and go back
   J: Forward until I'm on top and then back when on

    - Where would you apply this action in the physical world as well as the game world?
     - Why?
   MW: P: Scroll wheel G: The wheel
   M: P: Scroll wheel G: The larger wheel through the smaller wheel
   K: P: Mouse wheel G: The big wheel
   I: P: On the wheel G: The big wheel
   J: On the wheel G: Apply backwards on big wheel

    - In what direction would you apply this action in the physical world as well as the game world (along a path, counter-clockwise, upwards, etc.)?
     - Why?
   MW: P: Backwards G: Backwards
   M: P: Back and forth G: clockwise/counter
   K: backwards spinning G: spinning backwards to move forward
   I: forward then backwards
   J: P: Forward and then back G: The same

    - Is there something visually or audibly in the game world or the physical world that instructs you in how to apply this mechanic?
     - What?
   MW: Appearance of wheels
   M: Beam affords rolling. The entire contraption affords rolling
   K: Gaps!, big round wheel and the beam
   I: Thought the beams were square. The beam being round and the gaps
   J: The beam implies that it is the big wheel's axle and therefore I need to roll backwards on the big wheel

    - How much force/speed/acceleration do you think you need to apply in the physical world as well as the game world?
     - Why?
   MW: As fast as scrolling a webpage to get to the bottom
   M: Same in both. Considerate amount of force
   K: A little in the beginning to learn. If too fast I won't apply enough friction
   I: Quiet a bit to go up, on top I have to be careful
   J: The big thing looks heave so I would need to apply a lot of force to the small wheel, but not enough that it rolls off

    - Do you think you need to apply a certain kind of attitude when using this mechanic (aggressive/stealthily/carefully)?
     - Why?
   MW: P: aggressive. It translates through the worlds when going backwards
   M: Careful
   K: Patient
   I: Aggressive, then careful
   J: Precision, patience

  - Now that you have invoked the action, do you feel it was effective in overcoming the challenge?
   - Why?
   M: Yes
   K: Yes
   I: Yes
   J: Yes

    - Did you feel that the action and the effect happened synchronously?
    M: Yes
    K: Yes
    I: Yes
    J: Yeah sort of, felt a bit of lag. The big wheel didn't rotate as much as expected

    - Did you feel that the action and the effect happened at the same location?
    M: Yes
    K: Yes
    I: Yes
    J: A bit offset more on the side. But yes

    - Did you feel that the direction of the action was mirrored in the direction of the effect?
    M: Yes
    K: Yes, but opposite
    I: Mostly, yes. If I was not pointed straight I would go toward the edge
    J: Directly opposite so yes

    - Did you feel that there was any visual, tactile or audible feature that arose as a direct consequence of you action?
    M: The rolling of the small wheel yes
    K: Yes, visual, no feedback on a tangible aspect. Visual: The big wheel moved.
    I: Not tactile, physically I felt the rotation of the wheel, but there was nothing vibrating. Visual, yes it rolled, I was surprised by the mass of the small wheel
    J: The small wheel has small spikes that make it seem grip-like, which implied that it would stick more to the surface. The big wheel and the beams started spinning

    - Did you feel that the amount of force/speed/acceleration from your action was apparent in the effect?
    M: Not in the big wheel because of the small beams, expected to roll on ground if it were on the ground it would translate the force 1:1
    K: Yes, apparent. If I spin too fast I would fly off.
    I: Yes, the faster I rolled the faster it rolled, but I mostly used my weight.
    J: Yes, the thing started rotating in the opposite direction at about the same speed as the small wheel

    - Did you feel that the attitude you applied in your action was apparent in the effect?
    M: Yes, very much
    K: Yes, both when attitude was unpatient and patient.
    I: Yes, a bit of aggression and relaxing worked.
    J: Yes, oh yes.


 Questions:
 Spørgsmålene gav mening, men der var mange af dem der ikke var relevante i den konkrete situation. Desuden savnes mere uddybende svar. Forslag er at droppe urelevante spørgsmål og så gå i detaljen med de urelevante og tilføje spørgsmålet 'Did your action invoke the consequences you expected?'

 Framework gav mening som sketchingværktøj. Vil helst ikke gå i højere detaljer da det bryder med det essentielle for sketching. Forslag er at man supplerer med notetagning og pen and paper sketching. Desuden kan man visuelt sammenligne forskellige løsningsforslag, hvilket vil være brugbart.

 Suggestions:
 Coming into the room from a different angle to better see the roundness of the wheel and the beams.
 Make the beam appear more round.
 Delete the beam and place the big wheel on the ground and add groove under the big wheel to inform of its capability of going back and forth.
