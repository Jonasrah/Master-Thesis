- Conceptual models in games can be the game world or the theme of the game. e.g. a game set in an old Aztec-like temple would give the player a mental model of there possibly would be residing traps and puzzles, like they have learned from for example Indiana Jones. Ask Hajo for lit on game world that can
help this argument. Totten. Rollings

- Keyboard Sport uses natural mapping

- Norman's natural mapping and best, second-best and third-best mapping is not connected in any way. It is unclear what the difference between a good utilisation of mapping and natural mapping and since 'best mapping' is where controls are mounted directly to the controlled, natural mapping only qualifies as either second-best or third-best.

- The way the scroll wheel is interacted with becomes a carrier of meaning in the sense that one can scroll rapidly to gain speed or slow for precision.

- Action-perception loop == Feedback loop?

- Logical constraints can provide coupling: A room with a door and a button. Logical conclusion: Button operates door.

- Framework can also be used to decouple unintended action-perception loops.

 - Affordance addition: "An affordance is a three-way relationship between the environment, the organism, and an activity" \cite[p. 118]{dourish}.

 - Dourish and meaning:
  - A user's ontology == a user's conceptual model (p. 129).


Limitations:
 - Complexity of computer systems: "The consequence, then, is that there are very many different levels of description that could be used to describe my activity at any given moment. Some, perhaps, are ready-to-hand and some present-at-hand at the same time; my orientation toward them each will change. For instance, sometimes as I move the mouse, the mouse itself is the focus of my attention; some-times I am directed instead toward the cursor that it controls on the screen; at other times, I am directed toward the button I want to push, the e-mail message I want to send, or the lunch engagement I am trying to make" \cite[p. 140]{dourish}.
