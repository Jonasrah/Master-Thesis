\section{Into the Void}
The Frogger Framework is not limited to electronic products: In \citeA{frogger} a mechanical product is analysed through the framework and in \citeA{transbehav} it is explained how a speed-skate experience is augmented as a result of using the framework. The limit of the framework is, perhaps purposefully, left unclear and its use is described as ``[...] to explore and create intuitive and aesthetic interaction'' \cite[p. 22]{transbehav}. Also, the initial formulation of feedforward and inherent feedback was done with the believe that meaning is of highest priority when designing: ``We argue that the creation of meaning in interaction is the key to making abstract concepts in consumer electronics accessible'' \cite[p. 2]{howdonald}. This concern of meaning is shared in the game design community: ``The goal of successful game design is the creation of meaningful play ''\cite[ch. 3, p. 3]{salen}. This conclusion is furthermore derived from \cite{huizinga}: ``In play there is something 'at play' which transcends the immediate needs of life and imparts meaning to the action. All play means something'' \cite[p. 1]{huizinga}. The difference between the approaches would be the use of the words \textit{interaction} and \textit{play}. With respect to differing and expansive definitions of what play is \cite{huizinga, sicartplay, salen}, I will regard it within the narrow scope of \textit{game play}, since I am operating in the field of game design. With this in mind, play can be regarded as a form of interaction: ``Game play is the formalized interaction that occurs when players follow the rules of a game and experience its system through play'' \cite[ch. 22, p. 3]{salen}.
Returning to the Frogger Framework: While the use of the framework may be unclear, it is clear that \citeA{frogger} operates within the field of \textit{embodied interaction} \cite{transbehav}. A possible intention for use of the framework can therefore be regarded as within the field of embodied interaction. Following previous notions, embodied interaction is also defined as being dependant on meaning \cite[p. 126]{dourish}. Although similar interests of meaning is prevalent in designing for games \cite{salen} and embodied interaction \cite{dourish}, an assumption that this would make the two correlative would be frail. So, a further analysis of embodied interaction is needed.

\subsection{Embodied Interaction}
Since embodied interaction relies heavily on the definition of \textit{embodiment}, I will shortly delve into the phenomenological tradition of Husserl, Heidegger and Merleau-Ponty, all of whom helped better the understanding of embodiment, as deduced by \citeA{dourish}. To ensure focus, I will adhere to \citeauthor{dourish}'s \citeyear{dourish} analysis of the three phenomenologists' work.

\paragraph{Husserl} Edmund Husserl was the first to define phenomenology. He found that current sciences of his time (1859-1938) was diverting from the physical everyday world with practical appliances into an idealised one which was not experienced but only theorised. A student of Franz Brentano, he embraced Brentano's notion of \textit{intentionality} which describes the relationship between the external reality and one's thoughts about it, i.e. the relationship between the chair you might be sitting on and your thoughts about it. Husserl imagined phenomenology as a rigorous science for examining the nature of intentionality. For this methodology he terms (1) \textit{noema} and (2) \textit{noesis} as being representations of (1) objects of intentionality and (2) our mental experiences of those objects. For this methodology to be rigorous Husserl defines the objective of phenomenology as to explain how we perceive objects as simply existing without question, i.e. explaining how a \textit{natural attitude} arises. This world of natural attitudes, Husserl later defines as the \textit{lebenswelt} (or \textit{life-world}). ``The life-world is the intersubjective, mundane world of background understandings and experiences of the world. It is the world of the natural attitude and of everyday experience'' \cite[p. 106]{dourish}. Husserl further argues that there is a parallelism in the objects one perceives and the act of perceiving them, as when an object is perceived, the act of perceiving it is also accepted. Otherwise, real-world perceptions would be indistinguishable from imaginings. This also means that fantasising or memorising can be individual experiences.

Lastly, Husserl makes a distinction between the immediate everyday sense-impressions and the recognition of the object in front of us as being a chair. The distinction is deduced as moving from the everyday world towards the formal world. Recognising the object as a chair is recognising the object's \textit{essence} \cite{dourish}.

\paragraph{Heidegger} Martin Heidegger, a student of Edmund Husserl, departs from Husserl's mentalistic approach. While Husserl adopted Descartes' \textit{cartesian dualism}, which dictates that ``[...] we 'occupy' two different and separate worlds, the world of physical reality and the world of mental experience'' \cite[p. 107]{dourish}, Heidegger was concerned with the lack of attention from Husserl and his peers to the world of physical reality and argued that being comes before thinking: In order to think one has to be. Therefore, Heidegger rejected this dualism and argued that being and thinking is intertwined and cannot be separate from the other. Heidegger's development of phenomenology was roughly a shift from an epistemological approach to an ontological one, i.e. instead of focusing on how we obtain knowledge of the world, we should focus on how the world presents itself to us: ``From his perspective, the meaningfulness of everyday experience lies not in the head, but in the world'' \cite[p. 107]{dourish}.

Intentionality, as Husserl defined it, was for Heidegger too passive. For Heidegger, intentionality is not only concerned with perception, but is of a practical nature. It is based on our concern for utilisation and how we can manipulate something, i.e. a chair is a chair because we can sit on it. This meaning, according to Heidegger, is already existing in the world and is only interpreted by us, and this interpretation is what Heidegger is concerned with.

From Heidegger's work, two concepts are of most importance for the context of this thesis: (1) \textit{Dasein}, (or \textit{being-in-the-world}) and (2) \textit{das Zeug} (or \textit{equipment/tools}) \cite{heidegger}.
(1) Dasein explains the experience of being in the world, and sometimes as exemplified in an entity experiencing the world as we do. Specifically it is the relation between the being and the world, which in turn are inseparable. For Heidegger, Dasein is active and with a practical approach. The world is perceived as objects to be used or manipulated, but also a medium through which to act. More concretely, Dasein often perceives the world with the intention of accomplishing goals.
(2) For Dasein to accomplish these goals there are Zeug. Heidegger sums up Zeug as being something in-order-to, meaning that the Zeug is always linked to a task. While being linked to a task, Zeug is also linked to other Zeug in ways that it ``[...] relies upon, works with, suggests, is similar or dissimilar to, or is otherwise related to other equipment'' \cite[p. 109]{dourish}. Heidegger's Zeug also plays a particular important role within the context of this thesis with his distinction between \textit{zuhanden} (ready-to-hand) and \textit{vorhanden} (present-at-hand) \cite{heidegger}. To explain the distinction I will exemplify it with you considering a regular pencil as the Zeug. You pick it up and position it between your fingers. As you align the pencil to your preference, the pencil is vorhanden: You have to account for weight and balance it, hence its presence is obvious to you. When you have positioned the pencil you start drawing. Quickly, the pencil is absorbed into your being, it becomes zuhanden: You no longer actively consider the weight or balance of the pencil, but instead you act through the pencil producing pencil strokes. In a sense, you have become a being capable of drawing, instead of a being holding a pencil. To elaborate, Heidegger suggest that the pencil is only objectively present to us because of its possibility of becoming vorhanden, and as soon as it becomes zuhanden it withdraws from the world as an individual object, and is instead incorporated in the being. Should you perhaps find this absurd and check your hand to affirm that the pencil is indeed still there, you will find that the pencil has returned to the state of vorhanden, and is therefore present independent of your being \cite{dourish}.

\paragraph{Merleau-Ponty} Maurice Merleau-Ponty was particularly concerned with the body's role in perception, and his work is central when discussing embodiment. For Merleau-Ponty the body was the key to bridging the work of Husserl and Heidegger, which he thought was addressing the same problems but from different perspectives \cite{ponty}. Merleau-Ponty argued that the body can be considered as more than just psychophysical: That, in relation to the senses of the body, it can be considered the medium through which the external reality is perceived. The body therefore is what connects Husserl's ideas of perception and Heidegger's concern with being-in-the-world (the external reality). This, in turn, means that the body is both subject and object: The perceiver's body being subject, and the perceived body as being object.

This addition of a body as a mediation between Husserl and Heidegger makes the importance of what embodiment means clear. For this, Merleau-Ponty offers three definitions: ``The first is the physical embodiment of a human subject, with legs and arms, and of a certain size and shape; the second is the set of bodily skills and situational responses that we have developed; and the third is the cultural 'skills', abilities, and understandings that we responsively gain from the cultural world in which we are embedded'' \cite[p. 114]{dourish}.
\\
Building on these four perspectives on embodiment, \citeA{dourish} draws a definition: ``Embodiment is the property of our engagement with the world that allows us to make it meaningful'' \cite[p. 126]{dourish}. With this definition he then goes on to define embodied interaction: ``Embodied Interaction is the creation, manipulation, and sharing of meaning through engaged interaction with artifacts'' \cite[p. 126]{dourish}.

Then, I return to the question of whether or not game design can be to design for embodied interaction. \citeA{dourish} believes that this is not the case: ``Even in an immersive virtual-reality environment, users are disconnected observers of a world they do not inhabit directly. They peer out at it, figure out what’s going on, decide on some course of action, and enact it through the narrow interface of the keyboard or the data-glove, carefully monitoring the result to see if it turns out the way they expected. Our experience in the everyday world is not of that sort'' \cite[p. 102]{dourish}. \citeA{dourish} argues that video games uses the physical world as a metaphor for interaction to convey meaning to the player, and that this is not compatible with the embodied interaction we encounter in the real world. While not directly disagreeing, I will provide a perhaps more nuanced argument on how talking about embodiment in video games can be constructive.

\subsection{Embodiment in Video Games}
As previously discussed, intentionality describes the relationship between action and meaning, i.e. the intention of an action. Now, imagine playing a video game with a 3D simulated world wherein you come upon a challenge of pressing a button to open a door. You manoeuvre your avatar up to the button and press it, with the intention of opening the door. For the door to be intentionally opened by you, a chain of couplings \cite{dourish} need to be made: First the input of your finger on the relevant key of the device running the game, then the in-game agent's push of the button and finally the door opening. You do not act directly on the door, but your intention is still directed towards the door.

This notion of coupling is not dissimilar to the one discussed in \citeA{frogger} albeit this definition from \citeA{dourish} is not confined to the six aspects of time, location, direction, dynamics, modality and expression. Instead, \citeA{dourish} argues that when you experience a tool as Heidegger's zuhanden, the tool and you are coupled. Furthermore, as is clear from the previous example, the coupling can cover distances and thereby be remote from a local action, a fact that is also addressed in \citeA{frogger} when discussing coupling on the aspect of location: ``One can argue that the location of your fingers and your hand do not coincide with the cut of the paper. But because the scissors become an extension of your hand when cutting the paper (Heidegger’s concept of ‘ready-to-hand’ or ‘zuhanden’), you act through the scissors at the same location as the paper is being cut'' \cite[p. 2]{frogger}. So, when a tool becomes zuhanden, local actions can have remote intentions.

Combining this observation with the fact that couplings can also be made with abstract representations \cite{dourish}, there is grounds to explore whether or not input devices for video games can be considered tools able to become zuhanden. Video game controllers have a tendency to be more complex than e.g. the pen from a previous example, and if you were to hand such a controller to a relative who had never used such a device before, there would be little chance that the controller would become zuhanden for him or her in the course of an afternoon. How, if possible, does such a device become zuhanden?

\subsubsection{Involvement}
\citeA{calleja} argues that \textit{involvement} is the key prerequisite to presence in video games. On the topic of zuhanden, \citeauthor{calleja}'s \citeyear{calleja} definition of in-game kinaesthetic involvement is of clear relevance. Kinaesthetic involvement relates to the way a game translates input from the player to the allowed \textit{agency} within the game context through an \textit{avatar} or \textit{miniatures} \cite{calleja}. Agency within games pertains to the effect the player is able to assert in the game context. It is not referring to any intentions the player may have, but is instead the ability to perform intention-carrying actions. The avatar and the miniature are one or more entities that carry out the actions dictated by the player. The distinction between the two is that the avatar works as a representation of the player in the game world and is operated independently. Kinaesthetic involvement, then, is the involvement required to control an avatar or miniature \cite{calleja}.

The varying amount of conscious effort kinaesthetic involvement may take up is dependant on the player's attention. A new player as an example may require a large amount of attention to the controls, and as she progress, less attention is needed, in turn minimising the effort required for kinaesthetic involvement. \citeA{calleja} describes the state of minimal or no attention towards involvement as the involvement being \textit{internalised}. This diminishing attention is similar, if not identical, to Heidegger's notion of equipment withdrawing from the world to become zuhanden: ``The internalization of kinesthetic involvement describes a situation in which the controls are learned to such a degree that the on-screen movement of avatars and miniatures feels unmediated'' \cite[p. 68]{calleja} and ``[...] players must be able to 'think with their fingers' to the point at which the player feels like an extension of the game or the game feels like and extension of the player'' (\citeA{morris} in \citeA[p. 68]{calleja}). So, admitting that the phenomenological discussion on the matter can be much more nuanced and complex, for the sake of keeping focus, I will from this point forward consider game controls on the same level as Heidegger's equipment, thereby also considering them as capable of becoming zuhanden.

Just as \citeA{dourish} then describes the embodied interaction of acting on a classical WIMP (windows, icons, menus, pointers) interface through coupling, I might do the same with video games and because of this alignment I would be able to use the Frogger Framework \cite{frogger}. This would, however, limit the intricate detail that can be present in e.g. 3D or VR video games. It is obvious that this detail is present in the game world, so addressing it would be paramount for my framework. How, then, to address this detail will be the topic of the next section.

\subsubsection{Ludic Subjectivity}
The notion of kinaesthetic involvement is, as discussed, part of the more general term of involvement and as internalised involvement intensifies and the six dimensions (kinaesthetic, spatial, shared, narrative, affective and ludic) start to blend, the experience of the player moves towards an experience of \textit{incorporation} \cite{calleja}. Incorporation is thusly defined as ``[...] the absorption of a virtual environment into consciousness, yielding a sense of habitation, which is supported by the systemically upheld embodiment of the player in a single location, as represented by the avatar'' \cite[p. 169]{calleja}.

\citeA{vellashort} believes that this embodiment of the player in the game world can be described as the \textit{embodied ludic subject} which implies a point within this world from which the player can relate: ``[...] the player’s incorporation in the form of the playable figure grants her a spatial standpoint within the gameworld – an origo or point of origin to which deictic terms like ‘here,’ ‘there,’ ‘ahead’ or ‘to the left’ relate'' \cite[p. 4]{vellashort}. This is in congruence with the notion of spatial involvement from \citeA{calleja}, which, like kinaesthetic involvement, implies a necessity of internalisation before the player can be incorporated. This means that an incorporated player inhabits a point within the game world through an avatar or miniature from which the player relates this internal environment including visual and auditory information.

Following the definition of embodiment, a being's capability to act upon the world is central \cite{ponty}. \citeA{vella} argues that this capability is determined by the representation of the player within the game world. The player's capability is thus defined and limited by e.g. an avatar. This sense of what 'I can' and 'cannot' do is what shapes the world around us. Remembering \citeA{heidegger}, the objects of the world is only present to us in the way that they are able to be acted upon by us. The fact that we, when stumped in solving a challenge in a video game, can counter 'I cannot make that jump' to a passer-by suggesting that we 'just jump', bolster the argument that what is occurring on a mental level of the incorporated player is not a 'I think that my avatar is able to' but an 'I can' \cite{vellashort}. Furthermore, just as embodiment requires an ability to act upon the world it also requires the reverse: For the world to act upon the body \cite{ponty}. \citeA{vella} terms games' ability to act upon the embodied ludic subject as \textit{passion}, but not in the word's popular definition, but as the inverse of the word \textit{action} \cite{vella}. The term concerns the way the game world or agents in it can act upon the ludic subject and in turn how the player experiences this possibility. When I, for example, sneak around in \citeA{metalgear} I embody the fear of being discovered.

Returning then, to the misjudgement of \citeA{dourish}: ``Even in an immersive virtual-reality environment, users are disconnected observers of a world they do not inhabit directly. They peer out at it, figure out what’s going on, decide on some course of action, and enact it through the narrow interface of the keyboard or the data-glove, carefully monitoring the result to see if it turns out the way they expected. Our experience in the everyday world is not of that sort'' \cite[p. 102]{dourish}. From this citation I can extract three arguments: (1) virtual game environments cannot be inhabited, (2) actions within a game world are mediated and require high attention and (3) experiences in the everyday world are dissimilar to the ones in a game world.

Through the research presented in this thesis I deem there to be substantial evidence countering these arguments. \citeauthor{dourish}'s \citeyear{dourish} first argument is countered by \citeauthor{vella}'s argument that the player's incorporation \cite{calleja} creates a spatial point within the game world from where the player inhabits said world. The second argument is countered by \citeauthor{calleja}'s \citeyear{calleja} concept of involvement (especially kinaesthetic involvement) and \citeauthor{vella}'s \citeyear{vella} further arguments for embodied ludic subjectivity. The third argument can thus be said to be misdirected since it can now be argued that the game world is incorporated into the player's lebenswelt and the playable figure becomes an extension of the player, in other words part of the player's \textit{body schema} \cite{vellashort} \cite{haans}. \textbf{UDBYG!!!!!!!!!!!!!!!!!!!}
