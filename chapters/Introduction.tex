\section*{Abstract}


\section*{Introduction}
In the field of Human-Computer Interaction (HCI) research has been conducted on the subject of what information is communicated before and after an interaction. A design framework for analysing person-product interaction, the Interaction Frogger Framework, has been proposed by \citeA{frogger} which has gained traction in the community \cite{transbehav, tangifrog, vermeulen, rthroughd, move, easy}. The framework promises that a unification between action and function through the coupling of six identified aspects results in intuitive interaction. In the field of game research and game design player interaction is of some consideration \cite{salen, fullerton, schell, adams, agustin}, but on the subject of comprehensive game-specific methods with intuitive game design as its focus, I have identified a need for furthering research. In this thesis, I will provide an adaptation of the Frogger Framework to a game-specific focus. I will ground the discussion in a rooted theoretical foundation and to accomplish this adaptation, I will argue against established limitations \cite{dourish} of what can be considered embodied interaction. In order to do so, I will attempt to position game design in relation to embodied interaction and examine how embodiment and playing games correlate. With the adapted framework, I will, furthermore, lay the groundwork for a method with the intent of providing game designers with a tool to design intuitive controls in video games. Finally, I will provide a perspective of what has been achieved and what needs to be done to further the research on the subject in focus.

The thesis will be divided into five chapters. The first chapter will address the foundational theory that the Frogger Framework is developed from. The second chapter will explore the applicability of the Frogger Framework and the potential alterations that would need to be made. The third chapter will present the adapted framework and apply it in three different game contexts. The fourth chapter will use the adapted framework as a springboard to propose a method for intuitive game design. The method will be tested by a group of experienced game developers with a prototype constructed for this purpose. The fifth and final chapter will provide a closing conclusion as well as a perspective on what is next for the framework and the method.
