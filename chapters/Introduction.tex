\section{Problem}
The research fields of feedback, feedforward and affordances have been explored within the context of interaction design since at least 1988, but within the context of games, research has been limited yet its application is apparent in games today. In the following thesis, I will cover existing theory within the fields of both game design and interaction design and bridge the gap between the two in the context of the terms feedforward and feedback. I will also apply my research to the creation of a prototype, which will require research of its own within the fields of game design and prototyping. For the prototype, I will also apply theory to the creation of puzzles since puzzles lend themselves well for researching usability because of their high complexity as a challenge. For the evaluation of the applied theory in the prototype I will conduct interviews and play-testing and for this, I will apply theory on testing and qualitative as well as quantitative data and how to acquire this data from the interviews and play-tests.

\section{Method}
Initially, a study of existing theory will form the basis of the formulation of the rest of progress. The status quo of multiple fields of research will be examined and an attempt at extracting a framework of how to best design for a tight coupling between action and function will be made. this will then lead directly to the next phase: Prototyping. By extracting the relevant knowledge in the area, a prototype will be created with the intent of applying previously examined theory. The prototype will consist of multiple challenges with two variations: 1) Using the best practice in relation to the previously extracted framework. 2) Using either a direct opposite of the previous framework or the framework with one or more altered elements. Once the prototype has been created, the applied theory within the prototype will be evaluated by conducting play-tests followed by interviews according to the status quo of this area. The results of the evaluation will then be analysed and form the basis for any conclusion.
