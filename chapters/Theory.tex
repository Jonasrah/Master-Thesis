\chapter{Theoretical Framework}

Meta

\section{Discoverability and Understanding}
\citeA{norman} identifies two properties of design: 1) \textit{Discoverability} and 2) \textit{understanding}. 1) The designs probability of an agent comprehending the possible and intended interaction. 2) The designs ability to make an agent understand why and how it works. For the design to have good discoverability it must communicate the proper information and only this. , and for good understanding it must learn the agent why the intended interaction works and perhaps more important why the unintended does not. These two properties are only achieved by applying six psychological concepts. For discoverability there are five distinct concepts: \textit{affordances, signifiers, constraints, mappings} and \textit{feedback}. For understanding there is the concept of \textit{conceptual models}. To fully unearth the properties of discoverability and understanding, I will clarify the definition of all of these six concepts. Before that, however, it is important to note that \citeA{norman} defines these concepts with the lens of how to utilise them to achieve good design, so that while the concepts are presented as universal to all design, the evaluation of how to utilise the concepts for achieving good design is subjective.
\paragraph{Affordances} These are the relationship between the design object and the agent interacting with it. Human-centred design (HCD) would therefore have affordances that describe the relationship between humans and a design object. As an example, a push and pull door may have a handle on one side and not the other: The handle affords a grasping of the handle by the agent, which in turn is necessary for the agent to be able to pull the door open. The side with no handle does not afford grasping. This is described as an anti-affordance, i.e. the prevention of an interaction. Without the affordance of grasping the affordance of pushing should persuade the agent. If there had been handles on both sides of the door - thereby communicating wrong information on one side - this would likely lead to an amount of confused agents who would pull where one should push, and would be an example of misleading discoverability.\cite{norman}. Affordances can be both perceivable and hidden MORE ON THIS.
\paragraph{Signifiers} Any affordance that cannot be perceived requires another type of communicating to make the agent aware of it. \citeA{norman} calls this a signifier. A signifier can be any perceivable indicator that instructs the agent of the possible and intended behaviour \cite{norman}. Staying with the previous example of a push and pull door: A typical example of a signifier being used in this context is the "push" or "pull" signs placed somewhere obvious on the door. These signs signify the intended interaction with the door. Signifiers can also be something more abstract, like footprints in the snow, signalling the preferred route from point A to B \cite{norman}.
\paragraph{Mappings} In mathematics, mappings are "[...] the relationship between the elements of two sets of things" \cite[p. 20]{norman}. In design, mapping is relevant when designing the layout of controls and displays. \citeA{norman} defines a good use of mapping in design as having spatial coherence between the layout of the controls and the elements that they control. In many modern 3D video games that uses a dual analogue controller, it has become standardized to use the mapping of left analogue stick moves the player's avatar in the tilted direction and the right rotates the view in the tilted direction. When designing for mapping one can use the techniques of 1) \textit{grouping} and 2) \textit{proximity}. 1) grouping together related controls and 2) placing the control near the element that it controls \cite{norman}. \textit{natural mapping} \cite{norman} is a concept used for describing a design "[...] where the relationship between the controls and the object to be controlled [...] is obvious" \cite[p. 115]{norman}. In the original Pokémon Blue Version game \cite{pokemon}, the player controls a character in the centre of the screen in a top-down perspective. The device it was made for, the Game Boy, has a four-directional button and these four directions are mapped naturally to the character's movement on screen (see figure \ref{pokemon}).

\citeA{norman} rank the effectiveness of mapping in three levels: 1) Best, 2) second-best and 3) third-best mapping. 1) The ideal mapping would be to have the controls situated directly on the controlled. 2) The next best would be to situate the controls near the controlled. 3) The least effective would be to mimic the spatial layout of the controlled in the controls \cite{norman}. Natural mapping would reside in the third level, which means that even the least effective mapping method could still result in an intuitive layout of controls.

\paragraph{Constraints}
\begin{wrapfigure}{i}{0.35\textwidth}
  \includegraphics[width=0.35\textwidth]{Pokemon}
  \caption{Pokémon Blue Version playing on a Game Boy}
  \label{pokemon}
\end{wrapfigure}
\citeA{norman} defines constraints as being four-part: 1) Physical, 2) cultural, 3) semantic and 4) logical. 1) The physical limits of a design, constrain the possible interactions with it \cite{norman}. Consider a regular scissor: It has two holes for inserting fingers. One is large and the other smaller. The smaller hole constrain the amount of fingers that may be inserted into it, thereby signifying the proper interaction method. The regular scissor is also an example of how physical constraints can be useful in preventing incorrect actions.

2) Culture provides a set of written and unwritten rules which affects the way a person may interpret a situation. A person from Greece may greet a friend differently than a person from Denmark. In design, these guidelines also act as constraints \cite{norman}. As an example, the difference in right- and left-hand traffic. This cultural constraint has led to cars in Japan, a left-hand traffic country, being designed with the driver's seat in the left side, which is the opposite in Germany, a right-hand traffic country. A Japanese tourist renting a car in Germany, without knowing this cultural difference, would walk to the left side of the car to start the engine. In Japan, the constraint also affects the way people navigate on foot: Going up or down stairs, people keep to the left, which for a tourist from Germany may lead to some unsuspected physical constraining.

3) When using a regular claw hammer the situation constrain the interaction. If the user is about to hammer in a nail, the user is inclined to use the flat side of the hammer. If the user accidentally bends the nail and wants to extract it, the user is inclined to use the claw. This is a semantic constraint. \cite{norman}.

4) Logical constraints are closely connected to the concept of natural mapping \cite{norman}. In the case of the Game Boy and Pokémon as discussed earlier, the device provides five visible buttons, excluding volume and power, two of them are circular and grouped on the right side, another two have the appearance of slim lines and the fifth and final has a cross-shape pointing up, down, left, right. When faced with the problem of how to move the character, the logical constraint that no other button points in directions, should guide the player towards the cross-shaped button.
\paragraph{Feedback} Feedback describes the information that is relayed back when an action has occurred. In design, it can be utilised to let the user know that her input has been received and/or the consequence of the action. \citeA{norman} argues that feedback should be immediate and informative. Imagine hitting a button to turn on the light and having no other immediate feedback than the tactile sensation of the button being pressed in and springing back out. There is a delay between the button press and the light turning on, but you do not know this. You wait for a while for the light to turn on, but get impatient. You press again, confused, and still nothing happens since you have unbeknownst turned off the light before it was able to light up. This is an example where the feedback is not immediate and there is no way for the user to know if her input has been received. Take the same example, but swap the button with a switch and take away the delay. Now, you push the switch and get the immediate feedback of the switch being in an altered state and the light turning on. The switch is more informative than the button, since it can be in two different states, mimicking the light that can be either on or off, and the delay makes it impossible to have immediate feedback. Another aspect to keep in mind when designing feedback is that too much feedback can end up confusing the user \cite{norman}. Complex systems can accumulate an abundance of information and relay a great deal of this back to the user, which can result in overburdening and confusing the user. This means that design that deals with much information needs to prioritise and plan its feedback so that important trumps less important information and that all feedback is not presented in the same instance \cite{norman}.
\paragraph{Conceptual Models} A toaster is a device that many of us know how to operate. The common toaster is operated by pulling down a lever that lowers a slice of bread and exposing it to radiating heat. Suppose a toaster is delivered to a secluded tribe deep in the Amazon. It would be within reason to argue that the tribe would have no clue to what it was for and more importantly: How it works. A conceptual model is a simple explanation, that reside in our minds, of how something works. The reader of this would in most cases have a conceptual model of how a common toaster works, and would be able to use that understanding to operate said toaster. The residents of the aforementioned tribe would, however, not have such a conceptual model, and would therefore not be able to operate the toaster, disregarding the probable lack of electricity anywhere near them. A conceptual model is, however, not just a model of how something is used properly, but also a model of what makes the thing work. Staying with the toaster example, the reader will presumably know that the lever that lowers the bread, is coupled with a spring that is locked down when the lever is lowered sufficiently. Let us conceive of a situation where you lower the lever of a toaster and assume that the lever is lowered sufficiently, however, it is not. The lever springs back up, since the spring has not been locked down. With your conceptual model of the toaster it is assumed that you surmise that the lever was not lowered enough for the spring to be locked down, and you attempt the lowering once more, making sure to check for the tactile feedback of the locking mechanism. This is an example of a conceptual model providing the knowledge to evaluate a failed interaction and calibrate the approach to successfully carry out the intended operation. "Conceptual models are valuable in providing understanding, in predicting how things will behave, and in figuring out what to do when things do not go as planned" \cite[p. 28]{norman}. The question of how one obtains a conceptual model is then relevant. \citeA{norman} defines the \textit{system image} as the image from which one gain a conceptual model. The system image is perceived through all the available material describing the object: documentation, signifiers, the object's physicality, "what was told to us in the sales literature, by salespeople and advertisements, by articles we may have read, by the product website and instruction manuals" \cite[p. 31]{norman}. It is then the role of the designer to as closely as possible match her own conceptual model with the user's through the utilisation of the system image \cite{norman}.

\section{Purpose and Coupling}
With these concepts defined by \citeA{norman} there is a strong foundation for creating a design framework for my prototype, however, these concepts focus solely on communicating how something works: "You can figure out the scissors because their operating parts are visible and the implications clear. The conceptual model is obvious, and there is effective use of signifiers, affordances, and constraints" \cite[p. 27]{norman} but how does one realise that a scissor is for cutting paper? \citeA{norman} does not describe how a design can communicate its purpose. Furthermore, when \citeA{norman} writes that the fact that a scissor's operating parts are visible make the scissor intuitive, an interesting point is made, but is not further explored in \citeA{norman}. In \citeA{normanold}, the reason for what makes the scissors intuitive could be described by the concept of \textit{visibility}, but this concept seems to have been incorporated into the more general concept of signifiers. \citeA{howdonald} and \citeA{frogger} offer a more detailed explanation with their definitions of \textit{feedforward} and \textit{feedback} with their subdivisions \textit{inherent, augmented} and \textit{functional}. Below I will delve into their definitions of these two concepts in correlation with the concepts of \citeA{norman} because they themselves consider the concepts as building on top of the notions of \citeA{normanold}, and because Norman's concepts has been revised \cite{norman} ex post facto. A major revision to regard is the addition of the concept of signifiers and the alteration of the definition for affordances. \citeA{norman} notes about affordances that "alas, the term became used in ways that had nothing to do with the original" \cite[p. 13]{norman}. He argues that the design community understood the term as a property instead of a relationship when it needed a term to describe the addition of instructional elements. Therefore, signifiers were termed and affordances were made more concrete. The more broad understanding of affordances is therefore the one that is present in \citeA{howdonald} and \citeA{frogger}. However, the arguments and definitions are still relevant for the bolstering of my design framework, since the insights made, do not overlap directly with the old or new insights of \citeA{norman}, but instead expand the research field.

\subsection{Feedforward} Feedforward is the information an object gives before the interaction. This is given both by the way the object looks, but also the action that it requires \cite{frogger}. To further describe the concept of feedforward, \citeA{frogger} divide the concept into three subdivisions:
\paragraph{Inherent} Resembles perceivable affordances \cite{norman} in the sense that it communicates the action possibilities of the design, but also how this action can be made \cite{frogger}.
\paragraph{Augmented} Communication about the action possibilities or purpose of the action coming from an additional source \cite{frogger}. As an example, a lit button in an elevator communicates to the newly arrived passengers that the car will make a stop at the floor signalled on the button. The light inside or next to the button would be augmented feedforward. This, \citeA{norman} would also describe as a signifier.
\paragraph{Functional} In contrast to signifiers, that communicates how and where, functional feedforward communicates the general purpose of the design, i.e. why to do the action \cite{frogger}. Functional feedforward can utilise the concept of \textit{product semantics} \cite{semantics}, where the user's previous experience is relied upon for her to find meaning in the design. Additionally, as described in the scissor example from earlier, functional feedforward can describe the communication that happens when the operating parts are visible to the user \cite{frogger}.

In \citeA{howdonald}, the application of functional feedforward (at this point only described as feedforward) is further explained as follows: The designer must create meaning for the user, and for this there are two approaches: 1) The semantic approach and 2) the direct approach \cite{howdonald}. 1) The first approach relies on the user's previous experiences and apply that to inform the user of its own resemblance to that previous experience. It often benefits from the use of iconography and symbols \cite{howdonald}. This approach resembles signifiers, and a product of this approach could presumably be interpreted as both a signifier and feedforward. 2) The direct approach instead, relies only on bodily and perceptual skills and the intention being that meaning should be created in the interaction. Since perceptual and bodily skills is relied upon, the approach requires tangible interaction \cite{howdonald}. The approach utilises the concept of natural mapping because of the way the spatial layout of a design can communicate purpose of an action, but it also proposes similar methods of MUFFLE SOUND WITH THING.

For applying feedforward in a design, \citeA{howdonald} argues that a direct approach should be taken. It takes behaviour and action as its starting point. Instead of focusing on semantics where metaphors and resemblance is relied on, the authors instead believes that meaning is created in the interaction, therefore relying highly on perceptual and bodily skills. \cite{howdonald}.

\subsection{Feedback}
\paragraph{Inherent}
\paragraph{Augmented} This can for example be a displeased audio cue stemming from an attempted disallowed action, indicating that the action is not possible.
\paragraph{Functional}
Inherent feedback means that the performed action and the feedback it gives is closely coupled. The feedback of the action should act as an innate consequence. To design for inherent feedback one must consider four issues: "The action of the agent and the feedback of the product occur in the
same location. [...] The direction of the product's feedback is the same as the action of
the agent. [...] The modality of the product's feedback is the same as the modality
of the agent's action. [...] The product's feedback and the agent's action coincide in time."
Note: This is always the case with mechanical interaction. What they describe as mechanical Norman describe as best mapping.
\cite{howdonald}.






\newpage
