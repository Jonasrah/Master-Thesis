\section{Into the Void}
The Frogger Framework is not limited to electronic products: In \citeA{frogger} a mechanical product is analysed through the framework and in \citeA{transbehav} it is explained how a speed-skate experience is augmented as a result of using the framework. The limit of the framework is, perhaps purposefully, left unclear and its use is described as "[...] to explore and create intuitive and aesthetic interaction" \cite[p. 22]{transbehav}. Also, the initial formulation of feedforward and inherent feedback was done with the believe that meaning is of highest priority when designing: "We argue that the creation of meaning in interaction is the key to making abstract concepts in consumer electronics accessible" \cite[p. 2]{howdonald}. This concern of meaning is shared in the game design community: "the goal of successful game design is the creation of meaningful play" \cite[ch. 3, p. 3]{salen}. This conclusion is furthermore derived from \cite{huizinga}: "In play there is something 'at play' which transcends the immediate needs of life and imparts meaning to the action. All play means something" \cite[p. 1]{huizinga}. The difference between the approaches would be the use of the words \textit{interaction} and \textit{play}. With respect to differing and expansive definitions of what play is \cite{huizinga, sicartplay, salen}, I will regard it within the narrow scope of \textit{game play}, since I am operating in the field of game design. With this in mind, play can be regarded as a form of interaction: "Game play is the formalized interaction that occurs when players follow the rules of a game and experience its system through play" \cite[ch. 22, p. 3]{salen}.
Returning to the Frogger Framework: While the use of the framework may be unclear, it is clear that \citeA{frogger} operates within the field of \textit{embodied interaction}. A possible intention for use of the framework can therefore be regarded as within the field of embodied interaction. Following previous notions, embodied interaction is also defined as being dependant on meaning \cite[p. 126]{dourish}. Although similar interests of meaning is prevalent in designing for games \cite{salen} and embodied interaction \cite{dourish}, an assumption that this would make the two correlative would be frail. So, a further analysis of embodied interaction is needed.

\subsection{Embodied Interaction}
Since embodied interaction relies heavily on the definition of \textit{embodiment}, I will delve into the phenomenological tradition of Husserl, Heidegger, Schutz and Merleau-Ponty, all of whom helped better the understanding of embodiment, as deduced by \citeA{dourish}. To ensure focus, I will adhere to \citeauthor{dourish}'s \citeyear{dourish} analysis of the four phenomenologists' work.
\paragraph{Husserl}
\paragraph{Heidegger}
\paragraph{Schutz}
\paragraph{Merleau-Ponty}

From these four perspectives on embodiment, \citeA{dourish} draws a definition: "Embodiment is the property of our engagement with the world that allows us to make it meaningful" \cite[p. 126]{dourish}. With this definition he then goes on to define embodied interaction: "Embodied Interaction is the creation, manipulation, and sharing of meaning through engaged interaction with artifacts" \cite[p. 126]{dourish}.

"But because the scissors become an extension of your hand when cutting the paper (Heidegger’s concept of ‘ready-to-hand’ or ‘zuhanden’ [in 3, p.109]), you act through the scissors at the same location as the paper is being cut" \cite[p. 2]{frogger}

- Heidegger's Dasein (being-in-the-world), ready-at-hand, present at hand: \citeA[p. 107]{dourish}.

Present-at-hand and ready-at-hand

- Dourish's term embodiment does not encompass 3D games because users are disconnected observers of a world they do not inhabit directly.

So embodied interaction is not possible for the player, but what about the agent that the player controls? SHARED MEANING!!!

\subsection{Embodiment}
Haans and Ijsselsteijn
Three orders of embodiment.
\subsubsection{Morphology}
The body
\subsubsection{Body Schema}
The body and tools
\subsubsection{Body Image}
The body and tools as self.


\subsection{Involvement}
Calleja
Argument: The involvement becomes so intense that the kinesthetic aspect of inputting controls disappears.

\subsection{The Ludic I}
Vella
Games and telepresence
