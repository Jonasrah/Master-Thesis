The framework is divided in two: The physical world and the game world. All things only existing in the physical world operates on this side, and everything diegetic to the game world, i.e. what can be seen, heard, felt, tasted, smelled, touched, etc. by an agent in the game world operates on the game world side. All information transferred between these divisions is sent and/or received through the augmented information of the real world.

\subsection{Physical World}


\subsection{Game World}

\subsubsection{Inherent Information in a Game World}
Everything diegetic to the game world that uses inherent information (hills or tar that make you walk slower, elastics, wheels). Is highly tied to the conceptual model of the player, i.e. how does physics work. 

\subsubsection{Augmented Information in a Game World}
Everything diegetic to the game world that uses augmented information (stop signs, buttons, signifiers). It is not a HUD since that does not exist in the game world.

\subsection{Scope}
The framework cannot be used to analyse a game as a whole, but works better the smaller the focus. It can for example be used to analyse how a mechanic is conveyed EXAMPLE, how a mechanic is used, how a challenge is perceived, how a challenge is solved
