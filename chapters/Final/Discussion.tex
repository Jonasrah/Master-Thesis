\chapter{What Have I Done}
\textit{``Science isn't about why - it's about why not''} \\
\rightline{Cave Johnson} \\
\rightline{Portal 2 \cite{portaltwo}}

\section{Conclusion}
The goal of this thesis was to develop a method for creating intuitive interaction through game design. I started by exploring the foundational theories of \citeA{norman} and used this knowledge to delve deeper into the critique and expansion by \citeA{howdonald}, \citeA{frogger} and \citeA{transbehav}. This was done in order to ground myself in the field of HCI which is the field of the identified framework created with the intention of assisting the design of intuitive person-product interaction.

Following the work of \citeA{frogger}, I explored their proposed Frogger Framework with a desire to improve and adapt this framework to a context of game analysis and design. I did this on the basis of the desire to further the academic research in the area of intuitive game design and with the intention of adapting said framework to a game context.

This lead to an exploration of embodied interaction, the identified branch of design that the Frogger Framework is based upon \cite{frogger}, and whether or not video games share enough similarities to be compatible. \citeA{dourish}, the author behind the definition of embodied interaction, explicitly states that video games do not belong in the branch of embodied interaction, because of the disconnection between the body and the game world \cite{dourish}. Referencing \citeauthor{dourish}'s \citeyear{dourish} own notion of coupling, I voiced a possibility of being critical towards his argument. Then, being critical about his conclusion, I tested the credibility of his argument by exploring the rationale behind it, and at the same time analysing the nature of embodied interaction.

Exploring embodied interaction, this meant first exploring embodiment. For this, I positioned myself next to the phenomenological analysis of \citeA{dourish}, eventually reaching his definition of embodied interaction. Agreeing with the proposed definition, I argued against his exclusion of video games by referencing the same core phenomenological work of Husserl, Heidegger and Merleau-Ponty that he had analysed and the works of \citeA{vella} and \citeA{calleja}. It was clear to me that there were unavoidable similarities between Heidegger's concept of zuhanden and \citeauthor{calleja}'s \citeyear{calleja} concept of internalisation of controls. Combining this with \citeauthor{vella}'s \citeyear{vella} idea that a player's incorporation into the game world creates a point of origin, from where the player embodies the world, I finally made the argument that video games can be considered embodied interaction. This allowed me to establish a reworking of the Frogger Framework into a game context and to present a visual model.

The main difference from the Frogger Framework was presented as the distinction between the physical world and the game world. The distinction was made to address the detail that is to be found in interactions in the physical world and the game world and how they relate. The definition was first defined by the concept of diegesis \cite{bordwell} and was further developed by drawing on the concept of the ludic heterocosm as described by \citeA{vella}. What was central in the drawing in of the ludic heterocosm was the defining characteristic of it not just being the world represented on the screen visualising the video game, but instead is the world imagined in the mind of the player, thereby being more vast and inclusive.

Then, continuing with the delimiting of this new framework, I delved into the concept of kinaesthetic involvement from \citeA{calleja}, putting emphasis on the relation between attention and internalisation of controls in video games. Here, I was concerned with finding the appropriate intention for the framework. The Frogger Framework is mainly focused on product design \cite{frogger}, and so far I had only argued for the possibility of its use in a general game context. Looking back at what notions made the adaptation possible, I realised that the appropriate focus was the application of control from the physical world into the game world, which is also the focus of kinaesthetic involvement \cite{calleja}. With the intention of the Frogger Framework being to make intuitiveness of interaction clear \cite{frogger}, I too wanted the adaptation to have the intention of making controls clear, which \citeA{calleja} would describe as being concerned with minimising the attention of the player towards kinaesthetic involvement and ultimately the internalisation of controls.

Next objective was then to delineate the appropriate context for the application of the framework, and for this, I needed a rigidly defined language. With the objective of defining the appropriate application context of the framework I first wanted to explore in what setting the framework would be most constructive. I looked for a way of defining the setting in which control can be asserted. For this, I used the definition of kinaesthetic challenges as provided by \citeA{karhulahti}. Briefly summarising the definition, \citeA{karhulahti} describes kinaesthetic challenges as the required nontrivial effort at least partly necessitating the psychomotor skills of the player. This was identified as an appropriate setting since kinaesthetic involvement was identified as being implied in these types of challenges. Next, I looked for a definition of what I mean by asserting control. For this, I used the definition of game mechanics as methods invoked by agents, designed for interaction with the game state which was provided by \citeA{sicartmechanic}. Furthermore, I addressed the fact that if game mechanic is to be used in an analysis, it must not be considered a compound game mechanic, i.e. a game mechanic made up of several game mechanics, it should instead be broken up into the most basic form, as the complexity of analysing a compound game mechanic would lead to contradiction.

As an intermediate goal, I was able to propose and apply a comprehensive framework for analysing kinaesthetic challenges and game mechanics with results illustrating the intuitiveness of the context. With a framework for analysis, I set out to develop a method from said framework to be used in the context of game design. I identified the critical obstacle of this task to be the complexity of the framework, with it requiring an abundance of conscious effort. To solve this, I chose to divide the effort into the processes of playtesting and ideation. To test the constructiveness of the method, I gathered a group of possible users to act as both playtesters and designers and evaluate the method afterwards. The test revealed some problems with the structure of the interview during the playtest and some shortcomings in the preliminary instructions, both of which I have suggested possible solutions for. The overall result of the test did, however, show a desire from the participants to use the method in a design situation, revealing that the method can be constructive in designing for intuitive interaction in video games.

The proposed framework and method are both in their infancy and there is still work to be done. I consider this the inauguration of providing concrete methods for intuitive game design and hope for future research in this subject. Meanwhile, I end this thesis by identifying how the research presented here can be developed.

\section{Perspective}
As have been mentioned a few times by now, more needs to be done. The framework and the method should be applied in the intended situation; game analysis and game design, respectively. With the research presented here conforming to a mix of the practice of research as design studies \cite{rigour} and as research for design \cite{designtotheory}, future research could be with the approach of research through design, a research approach that is gaining followers \cite{rthroughd, designtotheory}. The relevance of this approach can be seen in its defining productivity: ``The work can result in the conceptual frameworks for design and evidence of the value of guiding philosophies for design. In addition, it can result in methods in support of conceptual frameworks and guiding philosophies'' \cite[p. 2894]{designtotheory}. What would be the relevant result to have as a research goal, in this case, would be to prove, improve or disprove the proposed method. The straightforward concrete way to go about this would be to design a game and use the proposed method.

With this encouragement, I will provide two ideas for future research to act as starting blocks. One is an incentive to explore the proposed in-depth interview for games research method presented in \citeA{grmethods} or similar interview techniques for a game context and combine relevant findings to the playtesting phase of the method. As was clear from the test of the method, the playtesting phase deserves critique and improvement, and this could be done by looking for inspiration in relevant methods in the area. The second idea is to explore supplementary frameworks and methods. The Mechanics Dynamics and Aesthetics (MDA) framework from \citeA{mda} is an example of a framework that can be used iteratively in a game design context. The authors explain how it can be used to tune a game's mechanics, dynamics and aesthetics \cite{mda}, a process that complements the proposed method's process. Another relevant value of the MDA framework is its possibility to be used as a lens for looking at a game for both game designers and game players, which could be complementary for the two phases of the proposed method. Another source for relevant methods and approaches can be found in \citeA{frontiers}, where methods for identifying game mechanics, elements and goals are presented. Certainly, inspiration can be found here to formalise the identification of game mechanics and kinaesthetic challenges with the intention of analysis through the proposed framework or to assist the playtester in identifying critical game elements for later scrutiny by designers.

Finally, what has not been addressed is whether or not the six aspects of time, location, direction, modality, dynamics and expression is even suitable for a game context. What I have found is that especially the aspects of modality and expression rely highly on interpretation. What I would wish for further research is that effort is put in concretising what is of value in these six aspects.
